% ----------------------------------------------------
% Theory Development
% ----------------------------------------------------

\chapter{Theory Development}\label{ch:theory-development}

\section{The Calculation of Salinity From Conductivity}\label{sec:salinity-conductivity-relationship}

Salinity meters that use electrical conductivity are commonly known as \glspl{ctd}.
As depth is a measurement derived from pressure, CTp is the preferred designation when performing calculations~\cite{lewis_salinity_definition_and_calculation_1978}, which allows for the conductivity of a sample of water to be denoted by $C(S, T, p)$ where conductivity is a function of salinity $S$, temperature $T$, and pressure $p$ which is the convention in oceanography~\cite{lewis_salinity_definition_and_calculation_1978}.
Pressure in the salinity equation is taken relative to sea level where $p = 0 dbar$ is equivalent to an absolute pressure of $P = 101\ 325 Pa$~\cite{ioc_teos_2010}.
Using decibars (dbar) for pressure is a common practice in oceanography as it is a unit of pressure roughly equivalent to one meter of water depth~\cite{seabird_dbar_to_depth_2024}.

The Practical Salinity Scale defines Practical salinity $S_p$ in terms of a conductivity ratio $K_{15}$, which is the conductivity of a sample of water at a temperature of $15\degree C$ and a pressure equal to one standard atmosphere divided by the conductivity of a standard potassium chloride solution at the same temperature and pressure.
The standard potassium chloride solution is $32.4356g$ of $KCl$ dissolved in $1.000kg$ of water, and when the ratio between the conductivity of a sample of water and the standard solution, or $K_{15}$, equals 1, the Practical Salinity $S_p$ is, by definition, 35.~\cite{ioc_teos_2010}

When $K_{15}$ is not equal to 1, the Practical Salinity $S_p$ can be calculated using the PSS-78 equation shown in \refeqn{eqn:pss-78-k15}.
\begin{equation}\label{eqn:pss-78-k15}
 S_p = \sum_{i=0}^{5} a_i {(K_{15})}^{i/2}~~~~\text{where}~~~~K_{15} = \frac{C(S_p, 15\degree C, 0)}{C(35, 15\degree C, 0)}
\end{equation}
All the coefficients for the salinity-conductivity equations, including $a_i$, are given in~\reftbl{tab:pss-78-coefficients}.

To calculate the salinity of a sample of water that is not at $15\degree C$ and $0 dbar$, the conductivity ratio of the sample can be expanded into the product of three ratios, which are labelled $R_p$, $R_t$, and $r_t$ respectively.
The conductivity measurement taken in the field $C(S_p, t, p)$ is related to the conductivity of the standard solution $C(35, 15\degree C, 0)$ which the device is calibrated with and is represented by $R$ in \refeqn{eqn:pss-78-full-ratio}.~\cite{ioc_teos_2010}
\begin{equation}\label{eqn:pss-78-full-ratio}
 R = \lfrac{C(S_p, t, p)}{C(35, 15\degree C, 0)} = \lfrac{C(S_p, t, p)}{C(S_p, t, 0)} \cdot \lfrac{C(S_p, t, 0)}{C(35, t, 0)} \cdot \lfrac{C(35, t, 0)}{C(35, 15\degree C, 0)} = R_p R_t r_t
\end{equation}
\textit{check}
In order to calculate the salinity of the sample, $R_t$ must be found, which takes a similar form to $K_{15}$.
$r_t$ is first calculated using the temperature of the sample 
\begin{equation}\label{eqn:pss-78-rt}
 r_t = \sum_{i=0}^{4} c_i {t}^i
\end{equation}
following which $R_p$ is calculated using the sample's pressure $p$, temperature $t$ and conductivity ratio $R$,
\begin{equation}\label{eqn:pss-78-rp}
 R_p = 1 + \lfrac{\sum_{i=1}^{3} e_i p^i}{1 + d_1 t + d_2 {t}^2 + R\left[ d_3 + d_4 t \right]}
\end{equation}
and finally $R_t$ is calculated using $r_t$, $R_p$ and $R$.
\begin{equation}\label{eqn:pss-78-Rt}
 R_t = \lfrac{R}{R_p r_t}
\end{equation}
For a sample temperature of $15\degree C$ and pressure of $0 dbar$, $r_t$ and $R_t$ both equal 1, which leaves $R_t$ equal to $R$ and thus \refeqn{eqn:pss-78-k15} can be used to calculate the Practical Salinity $S_p$.
For temperatures other than $15\degree C$, the Practical Salinity $S_p$ can be calculated using \refeqn{eqn:pss-78-full} where $k = 0.0162$.~\cite{ioc_teos_2010}
\begin{equation}\label{eqn:pss-78-full}
 S_p = \sum_{i=0}^{5} a_i {(R_t)}^{i/2} + \lfrac{t-15}{1+k(t-15)} \sum_{i=0}^{5} b_i {(R_t)}^{i/2}
\end{equation}

% chktex-file 44    
\begin{longtblr}[
 caption = {Coefficients for the PSS-78 equations~\cite{ioc_teos_2010}.},
 label = {tab:pss-78-coefficients}
 ]{
 colspec = {|q{1.5cm}|C|C|C|C|C|}
 }
    \hline
    \textbf{$i$} & \textbf{$a_i$} & \textbf{$b_i$} & \textbf{$c_i$} & \textbf{$d_i$} & \textbf{$e_i$} \\
    \hline
    $0$ & $0.0080$ & $0.0005$ & $6.766097\e{-1}$ & & \\
    \hline
    $1$ & $-0.1692$ & $-0.0056$ & $2.00564\e{-2}$ & $3.426\e{-2}$ & $2.070\e{-5}$ \\
    \hline
    $2$ & $25.3851$ & $-0.0066$ & $1.104259\e{-4}$ & $4.464\e{-4}$ & $-6.370\e{-10}$ \\
    \hline
    $3$ & $14.0941$ & $-0.0375$ & $-6.9698\e{-7}$ & $-4.215\e{-3}$ & $3.989\e{-15}$ \\
    \hline
    $4$ & $-7.0261$ & $0.0636$ & $1.0031\e{-9}$ & $-3.107\e{-3}$ & \\
    \hline
    $5$ & $2.7081$ & $-0.0144$ & & & \\
    \hline
\end{longtblr}
Note that the coefficients $a_i$ precisely sum to 35 such that the Practical Salinity $S_p$ is 35 when $K_{15}$ or $R_t = 1$ as per \refeqn{eqn:pss-78-k15} and \refeqn{eqn:pss-78-full}.
Additionally, the coefficients $b_i$ precisely sum to 0 such that the Practical Salinity $S_p$ does not depend on the temperature of the water when $R_t = 1$ as per \refeqn{eqn:pss-78-full}.~\cite{ioc_teos_2010}

\refeqn{eqn:pss-78-k15} to \refeqn{eqn:pss-78-full} are valid for $2 < S_p < 42$ and $-2\degree C < t < 35\degree C$ and $0 dbar < p < 10\ 000 dbar$~\cite{ioc_teos_2010}.
The range for salinity has been extended using estimations by \refref{hill_extension_of_pss_low_1986} for $0 < S_p < 2$ and \refref{poisson_extension_of_pss_high_1993} for $42 < S_p < 50$.

The temperatures used in \refeqn{eqn:pss-78-k15} to \refeqn{eqn:pss-78-full} are on the IPTS-68 scale~\cite{furukawa_ipts68_1973} and have not been corrected to the currently used ITS-90 scale~\cite{preston_its90_1990}. 
In order to correctly calculate the salinity, the temperatures should be converted to the IPTS-68 scale using the equation $t_{68} = 1.00024 t_{90}$ before calculating salinity~\cite{preston_its90_1990}.

% \section{Electrical Characteristics of Salt Water}\label{sec:electrical-characteristics-of-salt-water}
% PSU vs TSD vs conductivity vs resistivity, salinity equation, capacitance of salt water, non-constant conductivity vs voltage.

% \section{External Factors Affecting Electrical Characteristics of Salt Water}\label{sec:external-factors-affecting-electrical-characteristics-of-salt-water}

\section{Salt water's Resistance-Voltage Measurement}
% ostdiek_inquiry_into_physics_2020

\section{Current Fringing in Conductive Materials}

Current fringing or current spreading, not to be confused with magnetic fringing, occurs when an electrical current flowing through a conductive material spreads like a magnetic field.
This is a phenomenon that is particularly prevalent and well studied in the manufacturing of \glspl{led} where the current spreading can be a significant factor in the efficiency of the device~\cite{solomentsev_LED_current_spreading_2022, hwang_LED_current_spreading_2008, jeon_LED_current_spreading_2001}.

The effect of current spreading in typical conductors is mostly negligible; if the material's conductivity is high enough and its cross-sectional area is small, the current spreading is minimal.
However, if a conductor has a significantly larger cross-sectional area than the current requires, such as electrodes in salt water, the current spreading can become a significant factor in its conductivity.
This version of current spreading has been studied and is documented for conductors of constant conductivity.~\cite{jason_current_spreading_long_objects_2008}

% \section{Electromagnetic Interference of Salt Water}