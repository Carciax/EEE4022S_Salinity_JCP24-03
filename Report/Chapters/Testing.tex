% ----------------------------------------------------
% Methodology
% ----------------------------------------------------

\chapter{Salinometer Evaluation and Testing}\label{ch:testing}

The \gls{pcb} boards were delivered and tested.
Some design errors were found which include using op-amps that were rated for 6V instead of 3V3, missing a connection between VDDA and VCC, and footprints errors with both the temperature sensor and depth sensor.
The temperature sensor footprint was unable to be corrected, but the depth sensor could be corrected by flipping the depth sensor.

Once the circuitry was working and coded, the salinometer was tested.
The testing was conducted in two phases.
One phase before the probe was cast into epoxy resin and the other after the probe was cast into epoxy resin (section numbers?).
A summary of these tests is shown in~\reftbl{tab:testing-summary} and each test is discussed in further detail in the following sections.

The equipment used to verify these tests were a bench multimeter model Keysight U3401A which had voltage accuracy to $0.02\%$ and resistance accuracy to $0.1\%$.
% \textit{insert picutres, adc config, dac step, r^2 values}.

%chktex-file 44
\begin{longtblr}[
        caption = {A summary of the evaluation and testing of the salinometer.},
        label = {tab:testing-summary}
    ]{
        colspec = {|Q[c,m,1cm]|X[c,m]|*{3}{Q[c,m,1.75cm]|}}
    }
    \hline
    \textbf{Sec.} & \textbf{Test Description} & \textbf{Result Metric} & \textbf{Ideal Result} & \textbf{Measured Results} \\
    \hline
    \ref{sec:dac-voltage-range-and-accuracy} & The minimum and maximum voltage output of the \gls{dac} between $0V$ and $V_{DD} = 3.3V$ & Range & $0-3.3V$ & $0-2.59V$ \\
    \hline
    \ref{sec:dac-voltage-range-and-accuracy} & The gain and offset of the output voltage of the \gls{dac} relative to the instructed voltage & {Gain \\ Offset} & {$1.0$ \\ $0.0V$} & {$0.9837$ \\ $0.0070V$} \\
    \hline
    \ref{sec:adc-accuracy} & The gain and offset of the voltage measured by the \gls{adc} relative to the voltage measured by the multimeter & {Gain \\ Offset} & {$1.0$ \\ $0.0V$} & {$0.9877$ \\ $0.0082V$} \\
    \hline
    \ref{sec:calibration-resistance} & The resistance of the calibration resistor $R_{CAL}$ & Resistance & $5\Omega$ & $5.00\Omega$ \\
    \hline
    \ref{sec:resistance-measuring-accuracy} & The gain and offset of the resistance measured by the salinometer relative to the resistance measured by the multimeter & {Gain \\ Offset} & {$1$ \\ $0\Omega$} & {$1.0000$ \\ $0.0000\Omega$} \\
    \hline
    % Linear Conductivity Measurement Au & Whether the sample of salt water had linear conductivity throughout a range of voltages or not using the gold electrodes with no fringe shield & Non-linear &  \\
    % \hline
    % Linear Conductivity Measurement Shielded Au & Whether the sample of salt water had linear conductivity throughout a range of voltages or not using the gold electrodes with the fringe shield & Linear & \\
    % \hline
    % Linear Conductivity Measurement Ti & Whether the sample of salt water had linear conductivity throughout a range of voltages or not using the titanium electrodes & Non-Linear & \\
    % \hline
    % Titanium Voltage to Conductivity Mapping Accuracy & How accurately the relationship between the voltage output by the DAC and the voltage measured over the titanium electrodes can correlate to a specific conductivity & $100\%$ accuracy & \\
    % \hline
    % Salinity Measurement Accuracy & The salinometer's average accuracy in measuring salinity & $100\%$ & \\
    % \hline
\end{longtblr}

\section{DAC Voltage Range and Accuracy}\label{sec:dac-voltage-range-and-accuracy}

The \gls{dac} configuration uses a transistor in order to buffer the \gls{dac} output which allows for the power draw to be support by the transistor instead of the \gls{dac}.
This is a common configuration where the \gls{dac} is connected to the non-inverting input of an op-amp whose output is connected to the base of an NPN transistor.
The emitter of the transistor is then connected to the inverting input of the op-amp which allows the buffered output to match the input of the \gls{dac}.

This configuration does have one disadvantage in that the output voltage of the \gls{dac} is limited by the transistor's $V_{BE}$ such that the highest voltage output at the emitter of the transistor is $V_{DD} - V_{BE}$.
According to the transistor's \href{https://www.lcsc.com/datasheet/lcsc_datasheet_2310131500_Jiangsu-Changjing-Electronics-Technology-Co---Ltd--S8050-J3Y-RANGE-200-350_C2146.pdf}{data sheet}, the buffered output should be limited to $3.3V - 0.6V = 2.7V$ when conducting $0A$ and $3.3V - 0.75V = 2.55V$ when conducting the maximum current of $33mA$ when the load is $100\Omega$.
In order to assess the range and accuracy of the \gls{dac}, the \gls{dac} was instructed to output voltages from $0V$ to $V_{DD}$ in intervals of 64-bits and the output voltage was measured at the base and emitter of the buffer transistor and under maximum load of $100\Omega$ and no load.
$V_{DD}$ and $GND$ were measured to be $3.299V$ and $0V$ respectively.

\begin{figure}[!ht]
    \centering
    \begin{minipage}{.5\textwidth}
        \centering
        \includegraphics[width=\textwidth]{Figures/Testing/DAC_no_load}
        \caption{The input voltage versus the output voltage of the \gls{dac} with no load.}
        \label{fig:dac-voltage-range-no-load} %chktex 24
    \end{minipage}%
    \begin{minipage}{.5\textwidth}
        \centering
        \includegraphics[width=\textwidth]{Figures/Testing/DAC_loaded}
        \caption{The input voltage versus the output voltage of the \gls{dac} with a load of $100\Omega$.}
        \label{fig:dac-voltage-range-loaded} %chktex 24
    \end{minipage}
\end{figure}

The results were graphed and are shown in~\reffig{fig:dac-voltage-range-no-load} and~\reffig{fig:dac-voltage-range-loaded}.
The voltage drop as a result $V_{BE}$ can clearly be seen on~\reffig{fig:dac-voltage-range-loaded}.
The unloaded output voltage was able to reach $2.83V$ and the loaded output voltage was able to reach $2.59V$ which are slightly higher than the predicted limits.

An alternate attempt was also made to achieve a higher voltage output by using the internal reference voltage of the \gls{dac}.
The internal reference voltage was set to $4 \times 1.21V = 4.84V$ and the \gls{dac} was instructed to output the maximum voltage.
As expected, this was not able to increase the output voltage; the base of the transistor still outputted $3.3V$ and the emitter still outputted $2.83V$ while unloaded.

Due to the voltage limitations, the \gls{dac} will have a limited output in future testing and implementation to prevent the output voltage not reaching the desired input voltage.
The output will be limited to $0V$ to $2.5V$ or $0$ to $775$ for fully loaded tests and the implementation and $0V$ to $2.7$ or $0$ to $837$ for unloaded tests.
When excluding the voltage readings above $2.5V$, the \gls{dac} was able to achieve a gain of $0.9837V/V$ and an offset $+0.0070V$ between the input voltage and output voltage when under maximum load. 

\section{ADC Accuracy}\label{sec:adc-accuracy}

The \gls{adc} will be tested by measuring a range of voltages produced by the \gls{dac} and comparing the voltage measured by a multimeter to the voltage measured by the \gls{adc}.
The \gls{adc} will be configured in 12-bit mode with each measurement taking 15 \gls{adc} clock cycles and 5 measurements will be taken and averaged to increase the accuracy of the measurement.
The accuracy of the \gls{adc} should ideally be $100\%$.

\begin{figure}[!ht]
    \centering
    \includegraphics[width=0.5\textwidth]{Figures/Testing/ADC}
    \caption{The voltage output by the \gls{dac} measured by a multimeter versus measured by the \gls{adc}.}
    \label{fig:adc-accuracy} %chktex 24
\end{figure}

The results are shown in~\reffig{fig:adc-accuracy}. 
The \gls{adc} achieved a gain of $0.9877V/V$ and an offset of $0.0082V$ when compared to the multimeter.

\section{Calibration Resistance}\label{sec:calibration-resistance}

The calibration resistor will be measured by using the multimeter and by using the \gls{adc} with and without the gain applied.
The calibration resistance was specified to be $5\Omega \pm 0.25\%$, and thus it is expected to be between $4.9875$ and $5.0125\Omega$.

The calibration resistors were electrically disconnected, and the multimeter was used to measure the calibration resistor to be $5.25\Omega$ when the probes where applied directly across one of the parallel calibration resistor's terminals.
The multimeter cables measured $0.25\Omega$ when connected to each other and thus the final resistance of the calibration resistor was $5.00\Omega$.
It should be noted that the multimeter could only measure down to $0.01\Omega$ and thus the true resistance could range from $4.99\Omega$ to $5.01\Omega$.


\section{Resistance Measuring Accuracy}\label{sec:resistance-measuring-accuracy}

The method of measuring resistance involves getting a voltage reading of the calibration resistor and a sample resistor which is attached between the titanium electrode ports.
The resistance of the sample resistor is then calculated using the ratio between the voltage across the sample resistor and the calibration resistor.

This will be done using two methods: one with a single voltage from the \gls{dac} of $V_{DD}/2 = 1.65V$ and one with voltage sweep from the \gls{dac} with $50$ samples.
It was noticed during the testing phase that low voltage readings were not accurate as single bit errors caused large changes in the resistance reading and thus the range of voltages will be limited to $0.3V$ to $2.6V$ or $93$ to $806$ bits.
Both measurements will then be compared to the resistance measured by the multimeter.
The range of the resistors used will be $0\Omega$ to $10\Omega$ as this is the expected range for the gold electrodes.

\begin{figure}[!ht]
    \centering
    \begin{minipage}{.5\textwidth}
        \centering
        \includegraphics[width=\textwidth]{Figures/Testing/R_uncorrected}
        \caption{The resistance measuring test.}
        \label{fig:resistance-measuring-accuracy-uncorrected} %chktex 24
    \end{minipage}%
    \begin{minipage}{.5\textwidth}
        \centering
        \includegraphics[width=\textwidth]{Figures/Testing/R_corrected}
        \caption{The resistance measuring test using the corrected equation.}
        \label{fig:resistance-measuring-accuracy-corrected} %chktex 24
    \end{minipage}
\end{figure}

The results are shown in~\reffig{fig:resistance-measuring-accuracy-uncorrected}. 
The single voltage method and voltage sweep method were perfectly correlated with an $r^2$ value of $1.0000$, however there was a clear error between the actual resistance and the resistance measured by the salinometer.
This error was assumed to be due to the resistance of the switches and the traces.
While these values could be measured and corrected for, a more efficient and arguably more accurate method would be to generate a curve of best fit and use this to correct the resistance readings.

In order to generate the equation of best fit, the voltage ratio \refeqn{eqn:electrode-calib-resistance} is adjusted to include $r_e$ which represents the resistance of the switches and traces as shown in \refeqn{eqn:resistance-measuring}.
The $R_{calibration}$, $R_1$ and $r_e$ are condensed into the standard rational function coefficients $p$ and $q$ as shown in \refeqn{eqn:resistance-rational}.
Finally, the equation is rearranged to give the resistance of the electrode in terms the measured voltage ratio as shown in \refeqn{eqn:resistance-measuring-rational}.

\begin{align}
    V_{ratio} &= \lfrac{\lfrac{R_{electrode} + r_{e1}}{R_{electrode} + R_1 + r_{e2}}}{\lfrac{R_{calibration} + r_{e3}}{R_{calibration} + R_1 + r_{e4}}} \nonumber \\
    &= \lfrac{R_{electrode} + r_{e1}}{R_{electrode} + R_1 + r_{e2}} \times \lfrac{R_{calibration} + R_1 + r_{e4}}{R_{calibration} + r_{e3}} \label{eqn:resistance-measuring} \\
    V_{ratio} &= \lfrac{p_1 R_{electrode} + p_2}{R_{electrode} + q_1} \label{eqn:resistance-measuring-rational} \\
    R_{electrode} &= \lfrac{p_2 - q_1 V_{ratio}}{V_{ratio} - p_1} \label{eqn:resistance-measuring-rearranged}
\end{align}

The \refeqn{eqn:resistance-measuring-rational} of best fit was confirmed using \texttt{MATLAB} giving $p_1 = 17.4687$, $p_2 = 18.4643$ and $q_1 = 91.8315$ with an $r^2$ value of $1.0000$. 
The corrected resistance values were then obtained by applying \refeqn{eqn:resistance-measuring-rearranged} to the voltage ratios and these results were graphed and are shown in~\reffig{fig:resistance-measuring-accuracy-corrected} with a gain of $1.0000$ and an offset of $0.0000$.
Note that this correction equation is only valid when $R_1$ is $100\Omega$ and separate equations will need to be generated should different values of $R_1$ be needed.

\section{Linear Conductivity Measurement}\label{sec:linear-conductivity-measurement}

This testing phase happened after the probe was cast into epoxy.
The primary aim of this test is to discover if the salt water had a constant resistance for any given voltage.
The testing procedure for all the below tests was to configure the probe's electrode, sample count, \gls{dac} voltage range and anything else that may prove useful to experiment with.
A voltage sweep was then taken either forwards or backwards which was then sent over the \gls{uart} connection to a computer where it was processed using Microsoft Excel and MATLAB to generate the graphs and metrics.

In additional to the results mentioned below, these tests also demonstrated that the \gls{dac} input, output and the calibration voltages were all linear.
As mentioned above, in order to remove the linear gains of the op-amps, \gls{dac} and other components, only the ratio between the voltage across the electrodes and the calibration resistor voltage will be considered.

The first test aimed at discovering if the salt water had a linear resistance for any given voltage.
This involved using the gold electrodes with the fringe shield to isolate any potential sources of non-linearities.
These measurements were performed with both a forwards voltage sweep and a backwards voltage sweep.
Each voltage step measurement was also taken in both directions as mentioned earlier.
Ideally, these two measurements should be identical.
However, they were clearly not as shown in \reffig{fig:test-test1}.

\begin{figure}[ht]
    \begin{minipage}{0.5\textwidth}
        \centering
        \includegraphics[width=\textwidth]{Figures/Testing/Aus2}
        \caption{Conductivity test 1 with gold electrodes, the fringe shield, a voltage range of $0-2.6V$, and 50 samples taken of salt water sample of unknown .}
        \label{fig:test-test1} %chktex 24
    \end{minipage}
    \begin{minipage}{0.5\textwidth}
        \centering
        \includegraphics[width=\textwidth]{Figures/Testing/Aus3}
        \caption{Conductivity test 2 with draining and resetting the \gls{dac} between each sample.}
        \label{fig:test-test2} %chktex 24
    \end{minipage}
\end{figure}

In order to isolate the reason for the inconsistency, multiple tests were conducted with adjustments made between each test.
The first adjustment involved draining and resetting the \gls{dac} back to $0V$ each sample before it was set to the desired voltage.
It should be noted that this added a small delay between each sample of around $10ms$. 
While this did make the forward and reverse sweeps more similar as shown in \reffig{fig:test-test2}, the initial voltage lag of the reverse sweep on the right-hand side of the graph was still present.

This result brought forward an alternative idea that the measurements of the water briefly altered its properties.
This was unlikely to be a capacitive effect as the measurements were taken bidirectionally, and the capacitance would have been discharged.
It was theorized that the alternative current through the water could be ionizing the dissolved material, ripping electrons free and allowing them to move more freely.
However, this is purely speculative as the chemistry of the water is beyond the scope of this project and the knowledge of the author.

In order to further test this theory, the next test involved taking false, or priming, measurements at the maximum voltage before taking the actual measurements.
This caused the measurements in both directions to start high and slowly move to their predicted paths, as shown in \reffig{fig:test-test3}, which further supports the theory that the measurements were affecting the water.
This was further examined by using a varying number of priming measurements before conducting a reverse voltage sweep with 500 samples to accuracy track the voltage lag curve as shown in \reffig{fig:test-test4}.
It should be noted that the voltage measurement effective clips at $0.3V$ as the output of the $11\times$ gain op-amp reaches $0.3\times 11 = 3.3V$ which is the maximum voltage of the \gls{adc}.

\begin{figure}[ht]
    \begin{minipage}{0.5\textwidth}
        \centering
        \includegraphics[width=\textwidth]{Figures/Testing/Aus5}
        \caption{Conductivity test 3 with 25 priming measurements taken before the actual measurement.}
        \label{fig:test-test3} %chktex 24
    \end{minipage}
    \begin{minipage}{0.5\textwidth}
        \centering
        \includegraphics[width=\textwidth]{Figures/Testing/Aus7}
        \caption{Conductivity test 4 with a varying number of priming measurements and all true measurement taken as reverse voltage sweeps with 500 samples.}
        \label{fig:test-test4} %chktex 24
    \end{minipage}
\end{figure}

This concluded that while there is a potential for the measurements to be perfectly primed, it is unlikely that the priming affect will have zero impact on the final measurements.
It also indicates that the forward voltage sweeps could also be non-linear due to each subsequent measurement affecting the water before the next measurement is taken.
Thus, an alternative method was needed to take the measurements.    
Since the priming effect on the forward voltage sweep decreased to join the expected curve, it was theorized that the water could relax from a primed state over a short enough period of time.
More reasons needed here.
The next tests involved waiting for the salt water to settle between each measurement.

The first test involved waiting for $2s$ between each measurement and taking $50$ samples.
The relaxation waiting time happened while there was no voltage applied across the water.

\begin{figure}[ht]
    \begin{minipage}{0.5\textwidth}
        \centering
        \includegraphics[width=\textwidth]{Figures/Testing/Aus8}
        \caption{Conductivity test 5 with a varying amount of relaxation time before each measurement was taken and 50 samples.}
        \label{fig:test-test5} %chktex 24
    \end{minipage}
    \begin{minipage}{0.5\textwidth}
        \centering
        \includegraphics[width=\textwidth]{Figures/Testing/Aus10}
        \caption{Conductivity test 6 with 4 identical tests of 20 samples with 2s of relaxation time between measurements.}
        \label{fig:test-test6} %chktex 24
    \end{minipage}
\end{figure}



Initial tests showed that the measurement was non-linear for au electrode however forwards and backwards voltage sweeps are not the same GRAPH
this indicated that the measurement affected the voltage.

to isolate this it was thought that the dac might not be moving the voltage properly and thus the dac was reset and drained and then the voltage was pumped allowing for a longer time for the voltage to settle. 
this did not fix it. GRAPH

the next idea was to do false measurements at max voltage before doing sweeps.
this caused the readings to start high and slowly move to the votlage sweep. GRAPH
this shows that the measurement clearly affects the water by agitating it.
this could be ionising the water, or polarising it or another affected

the next test involved letting it settle between each measurement as in theory this would cause. 
starting at 2s.
this gave two graphs that matched perfecetly for the forward and backwards sweep. GRAPH

waiting 2s per measurement for 50 samples took very long and an alternate method was to change the water infront of the electrodes by agitating/stirring the water.
this gave repeatable results that also showed a difference between two different water samples. GRAPH
taking fewer samples over the more linear voltage range gave something that was approximately linear.

trying to find a decernable metric from these graphs by analysing A/(s+p) between two samples of arbitrary salinity. TABLE

try repeated single measurements.

ac testing not rigged up correctly. GRAPH