% ----------------------------------------------------
% Methodology
% ----------------------------------------------------

\chapter{Salinometer Evaluation and Testing}\label{ch:testing}

This chapter will discuss the evaluation and testing of the salinometer. 
A summary of these tests is shown in~\reftbl{tab:testing-summary} and each test is discussed in further detail in the following sections.
testing phase 1 and 2
floating point accuracy

%chktex-file 44
\begin{longtblr}[
    caption = {A summary of the evaluation and testing of the salinometer.},
    label = {tab:testing-summary}
    ]{
    colspec = {|C|C|C|C|}
    }
    \hline
    \textbf{Test} & \textbf{Description} & \textbf{Ideal Result} & \textbf{Measured Result} \\
    \hline
    DAC Voltage Range & The range of voltages that the DAC can output from $0V$ to $V_{DD} = 3.3V$ & $0-3.3V$ & \\
    \hline
    ADC Accuracy & The average accuracy of the ADC in measuring voltages from $DAC_{MIN}$ to $DAC_{MAX}$ & $100\%$ & \\
    \hline
    Calibration Resistance & The resistance of the calibration resistor $R_{CAL}$ & $5\Omega$ & \\
    \hline
    Calibration Resistance using ADC & The resistance of the calibration resistor $R_{CAL}$ as measured using the ADC & $5\Omega$ & \\
    \hline
    Resistance Measuring Accuracy & The salinometer's average accuracy in measuring specific resistances & $100\%$ & \\
    \hline
    Linear Conductivity Measurement Au & Whether the sample of salt water had linear conductivity throughout a range of voltages or not using the gold electrodes with no fringe shield & Non-linear &  \\
    \hline
    Linear Conductivity Measurement Shielded Au & Whether the sample of salt water had linear conductivity throughout a range of voltages or not using the gold electrodes with the fringe shield & Linear & \\
    \hline
    Linear Conductivity Measurement Ti & Whether the sample of salt water had linear conductivity throughout a range of voltages or not using the titanium electrodes & Non-Linear & \\
    \hline
    Titanium Voltage to Conductivity Mapping Accuracy & How accurately the relationship between the voltage output by the DAC and the voltage measured over the titanium electrodes can correlate to a specific conductivity & $100\%$ accuracy & \\
    \hline
    Salinity Measurement Accuracy & The salinometer's average accuracy in measuring salinity & $100\%$ & \\
    \hline
\end{longtblr}

\section{\gls{dac} Voltage Range}\label{sec:dac-voltage-range}

The range of the \gls{dac}'s data sheet that it is able to output is from $0V$ to $99.2\% V_{DD} = 3.27V$ which decreases with increasing load.
The ideal result of this test is that the \gls{dac} can output voltages from $0V$ to $3.3V$.
The \gls{dac} will be tested by outputting voltages from $0V$ to $3.3V$ and measuring the output voltage both before and after the buffer transistor.

The testing procedure for the \gls{dac} voltage range involves setting the \gls{dac} at a range of voltages between $0$ and $3.3V$ and measuring the output voltage before and after the buffer transistor both under maximum load of $100\Omega$ and no load.
The measurement equipment will be a Keysight U3401A multimeter as this is the most accurate multimeter available reporting accuracy to $0.02\%$. 
All measurement will be expressed as a percentage of $V_{DD}$.

The \gls{dac} was unable to reach the 3v3 required and even with the internal voltage set to 4x1.21 it was not able to output more than vcc
max = 2,8323V

\section{\gls{adc} Accuracy}\label{sec:adc-accuracy}

The \gls{adc} will be tested by measuring the voltage output by the \gls{dac} and comparing it to the voltage measured by a multimeter.
The ideal result of this test is that the \gls{adc} can measure the voltage output by the \gls{dac} with $100\%$ accuracy.

The \gls{adc} will be tested by measuring the voltage output by the \gls{dac} and comparing it to the voltage measured by a multimeter.
This range will be from 0 to DAC\_MAX which is 2.8 volts unloaded or 868/1024 bits.
The adc will take an average value of 5 samples to reduce noise and a delay will be added to allow the voltage to settle.

\section{Calibration Resistance}\label{sec:calibration-resistance}

The calibration resistor $R_{CAL}$ will be measured using a multimeter.
The ideal result of this test is that the calibration resistor is $5\Omega$. 
electrically disconnected from the circuit
resistance accuracy 0.1\%
result = 5,71 ohm
result direct = 5,62 ohm
wires = 0,50 ohm
final = 5,12 ohm

\section{Calibration Resistance using \gls{adc}}\label{sec:calibration-resistance-using-adc}

The calibration resistor $R_{CAL}$ will be measured using the \gls{adc}.
The ideal result of this test is that the calibration resistor is $5\Omega$.
The voltage that is expected to be measured across the calibration resistor is $1.729$ or $2146$ on a 12 bit \gls{adc}.
\begin{equation}
    V_{CAL} = \frac{V_{DD} \cdot R_{CAL}}{R_{CAL} + R_1} \cdot A_{op-amp} = \frac{3.3V \cdot 5\Omega}{5\Omega + 100\Omega} \cdot 11 = 1.729V = 2146 bits
\end{equation}

dac votlage set at VDD/2 @ 0x200. adc measurement averaged by 5 times.
vcc = 3,2930
dac = 1,6343
calib = 78,86 mV
calib = 96 bits
calib gain = 0,9413
calib gain = 1174 bits

running range of dac voltages from 0 to 2.6V (dac max loaded) = 155 to 806 bits to prevent weird occurrences.
measuring the calib voltage over the dac voltage. using adc.
test done while loaded with R100 + calib.


\section{Resistance Measuring Accuracy}\label{sec:resistance-measuring-accuracy}

The salinometer's accuracy in measuring specific resistances will be tested using a multimeter.
The ideal result of this test is that the salinometer can measure resistance with $100\%$ accuracy.
Several resistors will be measured by both the multimeter and the salinometer to determine the accuracy of the salinometer.

The salinometer will be calibrated for measuring zero ohms and the resistance of the resistors will be measured using a multimeter
The dac will run from 0.5V to 2.6V  (dac max loaded) = 155 to 806 bits to prevent weird occurrences. 