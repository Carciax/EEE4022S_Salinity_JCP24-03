% ----------------------------------------------------
% Recommendations
% ----------------------------------------------------

\chapter{Recommendations}\label{ch:recommendations}

Further investigation of the methods of conductivity measurements was required.
The voltage sweep method was vulnerable to individual measurements when adjusting the saltwater sample.
To avoid this, using a water pump to circulate the water during the measurement is recommended.
This setup would require a large volume of water of known salinity to be used, which would present additional challenges when calibrating the probe with the $35$ \gls{psu} standard solution at $15\degree C$ and $0dbar$.

The probe was not designed for measuring \gls{ac} voltages.
However, altering the probe to remove the capacitor that was likely interfering with the measurement would allow for a further investigation into \gls{ac} voltage's feasibility.
Additionally, a future iteration of the probe could be explicitly designed to measure \gls{ac} voltages and determine its feasibility.

The attachment of the gold electrodes to the probe \gls{pcb} was suboptimal, breaking on two occasions during the project.
An alternative method of attaching them would be using a slot joint~\cite{nicolaas_slot_joint_2015}, where a slot could be cut into the probe and the electrodes, allowing them to slide into each other and be soldered in place on both sides, which would potentially be more robust.
This design would make it challenging to have more than two connections to the electrode, but it could allow the probe to be thinner and taller, allowing it to fit down a smaller diameter ice core.

The controller proved successful in all its requirements, but a more user-friendly interface and housing would benefit the final device.
A housing could be 3D printed to make the device easier to handle, more aesthetically pleasing, and protect it from the elements.
Additionally, the 7-segment displays could be replaced with a \gls{lcd}, allowing for more detailed and clear information to be displayed to the user.
