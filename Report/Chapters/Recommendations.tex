% ----------------------------------------------------
% Recommendations
% ----------------------------------------------------

\chapter{Recommendations}\label{ch:recommendations}

Further investigation of the methods of conductivity measurements was required.
The voltage sweep method was vulnerable to the individual measurements adjusting the saltwater sample.
To avoid this, it is recommended to use a water pump to circulate the water during the measurement.
This would require a large volume of water of known salinity to be used, which would present additional challenges when calibrating the probe with the $35$ \gls{psu} standard solution at $15\degree C$ and $0dbar$.

The probe was not designed for measuring \gls{ac} voltages.
However, altering the probe to remove the capacitor that was likely interfering with the measurement would allow for a further investigation into its feasibility.
This would allow for a future iteration of the probe to be designed specifically to measure \gls{ac} voltages and determine its feasibility.

The attachment of the gold electrodes to the probe \gls{pcb} was suboptimal, breaking on two occasions during the project.
An alternative method of attaching them was proposed using a slot joint~\cite{nicolaas_slot_joint_2015}, where a slot could be cut into the probe and the electrodes, allowing them to slide into each other and be soldered in place.
This would also allow for the probe to be made thinner and taller, allowing it to fit vertically down a smaller diameter ice core.

The controller proved successful in all of its requirements, 

The controller proved successful, but in future iterations, a \gls{lcd} would create a more user-friendly interface.
Additionally, a housing should be 3D printed to protect the controller from the elements, and make the final device more aesthetically pleasing.
