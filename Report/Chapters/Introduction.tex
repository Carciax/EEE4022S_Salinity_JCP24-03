% ----------------------------------------------------
% Introduction
% ----------------------------------------------------

\chapter{Introduction}

\section{Problem Statement}
There are several methods and designs for measuring salinity.
However, none of these are ideal for the challenges faced when measuring the salinity of the ocean beneath the Antarctic ice sheet.
Antarctica is the coldest continent on Earth, covered in a vast sheet of ice consisting of around 30 million cubic kilometres in volume, which is about 60\% of the world's fresh water~\cite{NSIDC_ice_sheet_facts_2024}.
This ice sheet supports a variety of species both above and below it, but the ice itself was thought to be inhospitable to any form of life.
However, this has recently been disproven.
Small cracks, crevices and pockets in the ice create a habitat for several microorganisms that have adapted to its cold, harsh environment.
These microorganisms are the ecosystem's building blocks that survive in and beneath the ice sheet.

Scientists working with \gls{uct} have been studying this unique ecosystem and the physical properties of its habitat to understand how it can thrive in a previously thought lifeless environment.
Salinity is one of the physical properties of interest as it typically defines which species can survive in a given environment. \textit{CITE}.
The current method of measuring the salinity of the water beneath the ice sheet could be better for several reasons, and a new approach is needed, which this project aims to provide.

\section{Background}
The constant wax and wane of the Antarctic ice sheet creates a unique habitat that supports the microorganisms mentioned above.
The ice sheet thickens over time as water accumulates from precipitation, sea spray, and other sources.
Salts are also deposited with the water through atmospheric deposition and sea spray.
When the water freezes, the salt is expelled into small pockets and channels of highly saline water, known as brine channels.

The brine drains into the ocean, forming a mixture of frigid, saline water and normal seawater, creating a unique environment that could support life.
The properties of this mixture of water are not well understood, but they are of interest to scientists studying the ecosystem beneath the ice sheet.
Properties in the mixing zone, such as the water's salinity, temperature, currents, light penetration, and more, are all metrics that are currently being investigated.

To measure any given property, an ice core is drilled down to the water's surface, and two main methods are employed to measure the brine-sea water mixture.
Either a probe is lowered into the water through the ice core, allowing measurements to be taken at multiple depths, or a sample of the water is captured by lowering an open canister into the water, closing it at the desired depth, and retrieving it for analysis with hand-held instruments.
Salinity is currently being measured using the latter method, which is not ideal. The water sample changes temperature and pressure as it is brought to the surface, which can affect the salinity measurement.

\section{Objectives}
The ideal device is a salinometer with a probe that can be lowered down the ice core to measure the salinity of the water beneath the ice at various depths.
However, the aim of this project is to design a prototype, which would aid in investigating its feasibility and establishing a good understanding of the methodology for measuring salinity.
This prototype should set the foundation for a future device to be developed that can be used in the field.
While the final device needs to be able to perform in the harsh conditions and temperatures of Antarctica, these will only be secondary considerations of this project.

% \section{System Requirements}
% \lipsum[1]

\section{Scope \& Limitations}
This project's scope includes the design and development of a prototype device for measuring salinity.
This includes researching literature that details similar devices, the theory behind measuring salinity, and the design and development of a prototype device that can test the properties of salt water and lead to a method of measuring salinity.
The prototype development also includes testing and evaluating the device to determine its effectiveness in measuring salinity.
Additionally, this project should aim to develop the prototype as a separate probe and control unit.
The scope does not extend to any development for the final device beyond the prototype nor the analysis of any data captured should the prototype be used in the field.

This project has a budget limitation of R2000 for the entire design, development and testing.
This budget can only be spent through \gls{uct} with their approved suppliers and vendors.
The project must be completed in 14 weeks from the start to the submission of the final report.

\section{Report Outline}
\lipsum[1]