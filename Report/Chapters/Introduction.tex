% ----------------------------------------------------
% Introduction
% ----------------------------------------------------

\chapter{Introduction}

Antarctica is the coldest continent on Earth covered in a vast sheet of ice that contains about 30 million cubic kilometres of ice~\cite{NSIDC_ice_sheet_facts_2024} which is about 60\% of the world's fresh water.
This ice sheet has been considered a lifeless desert due to the harsh conditions and the lack of life that can survive in the cold, however, this has recently been disproven.
Hidden in the ice lies a vast ecosystem of microorganisms that has adapted to the cold and harsh conditions of the ice sheet.
These microorganisms are the building blocks of the ecosystem that thrives in the ice sheet.

The scientists working with \gls{uct} have been studying this unique ecosystem and the ice sheet itself to understand how it is able to thrive in a previously thought lifeless environment.
One of the properties of that needs to be analysed is the salinity of the ice sheet and the water beneath it.
The current method of measuring the salinity of the ice sheet is to drill ice cores and capture a sample of water to measure its salinity which is what this project aims to improve by creating a device that can be lowered into the ice core to measure the salinity of the water beneath the ice at various depths.

\section{Background}
The constant wax and wane of the Antarctic ice sheet creates a unique environment that habitats the microorganisms that live in the ice sheet.
The ice sheet is thicken by the accumulation of snow, ice and salt over time. 
When this mixture freezes, the salt is expelled from the ice and forms brine channels of highly saline water.
These brine channels then drain into the ocean and create the habitat supports the microorganisms and the ecosystems that feed off them.

When the brine channels drain into the ocean, they create a mixture of frigid, saline water and normal seawater.
This mixture of water is one of the topics of interest to the scientists as it is a black box of information that is not well understood.
How the water mixes, how the salinity changes through the depths, the currents, light penetration, temperature, how the environment supports life, and more are all matters to be studied.

The current method of analysing the salinity of the ice sheet is to drill ice cores to the surface of the water and either probe instruments into the water or capture samples of the water and measure its characteristics.
The method of capturing a sample of water is not destructive.
The latter method is not ideal as the sample changes as it is brought to the surface, and it would be better to measure the salinity of the water in situ which is what this project aims to achieve.

\section{Objectives}
The final device that this project aims to produce is a device that can be lowered into the ice core to measure the salinity of the water beneath the ice at various depths.
The device should be able to perform in the harsh conditions and temperatures of Antarctica and its ocean and should be able to achieve a good level of accuracy in its measurements.
The objectives of this report focus on the construction of a prototype of this device designed to investigate the feasibility of the project and to establish a good understand and methodology for measuring salinity.

\section{System Requirements}
\lipsum[1]

\section{Scope \& Limitations}
The scope of this project includes the design and development of a prototype device that is aimed at measuring salinity.
This includes research into literature detailing similar devices and the theory behind measuring salinity, the design and development of the prototype device that can test the properties of salt water and lead to a method of measuring salinity and finally the testing and evaluation of the prototype device and the properties of the salt water.
Additionally, the project should include the development of a method to control the device once it is deploying down the ice core.
The scope does not extend to the full development of a device that can be used in the field nor the gathering and analysis of data should this device be used in the field.

The limitations of this project include the fact that the device is a prototype and is not a final product.
The project has a budget limitation of R2000 for the full design, development and testing of the prototype device.
This budget can only be spent through \gls{uct} and their approved suppliers and vendors.
There is a time limitation of 14 weeks to complete the project from the start of the project to the submission of the final report.

\section{Report Outline}
\lipsum[1]