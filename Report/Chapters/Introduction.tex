% ----------------------------------------------------
% Introduction
% ----------------------------------------------------

\chapter{Introduction}

\section{Problem Statement}
There are several methods and designs for measuring salinity which is commonly understood as the salt content of water.
However, Antarctica's freezing temperatures and harsh conditions make it challenging for scientists to study this metric there.
The current method of measuring involves capturing a water sample from the ocean beneath the ice sheet and bringing it to the surface for measurement.
This process alters some water's physical properties, such as temperature and pressure, which can affect the salinity measurement.
This project aims to design a device that can more accurately measure the salinity by using a probe to take measurements at various depths in the ocean.

\section{Background}
Antarctica is covered in a vast sheet of ice consisting of around 30 million cubic kilometres in volume, which is about 60\% of the world's fresh water~\cite{NSIDC_ice_sheet_facts_2024}.
Part of the ice sheets rests on land, known as the continental ice sheet, while the rest floats on the ocean, known as the ice shelves~\cite{nsidc_ice_shelves_2024}.
The ice sheet supports a variety of species both above and below it, as well as an ecosystem within the ice itself~\cite{noaa_arctic_expedition_2002}.

The ice shelves surrounding the Antarctic continent constantly wax and wane~\cite{nasa_antarctic_sea_ice_2021}, with the ice thickening during winter with water and salt accumulating from precipitation, sea spray, and ocean water freezing in direct contact with the ice~\cite{hogg_extending_record_ice_shelf_thickness_2021}.
When the water freezes, the salts are expelled from the mixture,  creating small pockets and highly saline water channels known as brine channels.
The brine channels form a habitat for several microorganisms that have adapted to the cold, harsh environment and a new habitat when they drain into the ocean below.
Frigid brine and regular seawater mix beneath the ice shelves, creating a mixing zone which forms a unique environment that supports life.
Scientists working with the \gls{uct} are currently investigating the properties of the mixing zone, such as the water's salinity, temperature, currents, light penetration, and more.

To measure any given property, an ice core is drilled through the ice shelf down to the water's surface, and two main methods are employed to measure the brine-sea water mixture.
Either a probe is lowered into the water through the ice core hole, allowing measurements to be taken at multiple depths, or a sample of the water is captured by lowering an open canister known as a Niskin bottle into the water, closing it at the desired depth, and retrieving it for analysis with hand-held instruments.
Salinity is currently being measured using the latter method, which could be improved as the water sample changes temperature and pressure as it is brought to the surface, which can affect the salinity measurement.

\section{Objectives}
The objectives of this project are to design and develop a prototype device for measuring the salinity of the water beneath the ice shelf.
The prototype should aid in investigating its feasibility and understanding the methodology for measuring salinity.
It should set the foundation for a future device to be developed that can be used in the field.
The final device should be able to withstand Antarctica's harsh temperatures and conditions, but these will only be secondary considerations in this project.

% \section{System Requirements}
% \lipsum[1]

\section{Scope \& Limitations}
This project's scope includes the design and development of a prototype device for measuring salinity.
This includes researching the relevant literature that details devices with similar functionality, the theory behind measuring salinity, followed by the design and development of a prototype device that can test the properties of salt water and lead to a method of measuring salinity.
The prototype development involves testing and evaluating the device to determine its effectiveness in measuring salinity.
Additionally, this project should aim to develop the prototype as a separate probe and control unit.
The scope does not extend to any development for the final device beyond the prototype nor the analysis of any data captured should the prototype be used in the field.

This project has a budget limitation of R2000 for the entire design, development and testing.
This budget can only be spent through \gls{uct} with their approved suppliers and vendors.
The project must be completed in 14 weeks from the start to the submission of the final report.

\section{Report Outline}
\lipsum[1]