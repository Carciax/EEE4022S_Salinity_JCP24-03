\chapter*{Abstract}
The Antarctic ice shelves are a habitat that supports numerous unique species in and below the ice.
Melting ice causes salt to be expelled, which drains into the ocean beneath, creating a mixing zone with properties that are currently being investigated.
A prototype electrical conductivity-based salinometer was developed for this project to replace the current method of measuring the salinity of the mixing zone.
The prototype was designed with a separate probe, which would be lowered through a hole in the ice to the ocean below, and a controller, which would request and display data from the probe.
The probe utilised two types of electrodes made from gold and titanium to investigate the methods of measuring repeatable and accurate salinity measurements.
Voltage sweeps and \gls{ac} voltages could not produce accurate results as they needed further apparatus and investigations.
Single voltage measurements were able to calculate salinities to within $4$ \gls{psu} of the expected value but were vulnerable to noise and other errors.